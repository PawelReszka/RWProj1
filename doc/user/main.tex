\documentclass[a4paper]{article}

\usepackage[utf8]{inputenc}
\usepackage{polski}
\usepackage{amsmath}
\usepackage{graphicx}
\usepackage[colorinlistoftodos]{todonotes}

\title{Dokumentacja użytkownika implementacji projektu nr 1. RW}

\author{Tomasz Półgrabia}

\date{\today}

\begin{document}
\maketitle

\begin{abstract}

\end{abstract}

\section{Wstęp}
	Wstęp

\section{Wykorzystane technologie}
	Implementacja projektu nr 1. z RW będzie wykonana przy użyciu następujących technologii:
    \begin{itemize}
    	\item C\# (framework środowiska developerskiego .NET v4.5),
        \item prolog (implementacja SWI-prolog).
    \end{itemize}
    
    C\# zostanie wykorzystany w celu stworzenia wizualizacji zdarzeń zachodzących w czasie wykonania
    programu, natomiast prolog do implementacji logiki języka. Następnie zostanie wykorzystany
    interfejs C\# - prolog w celu zapewnienia komunikacji pomiędzy wyżej wymienionymi modułami.
    
    Do komunikacji pomiędzy modułami zostanie wykorzystana biblioteka SWIPlCs.
    
    \subsection{C\#}
    	TODO
    
    \subsection{Prolog}
    Prolog zostanie wykorzystany w projekcie \textit{TODO} w celu
    zaimplementowania logiki języka z domyślnymi akcjami. Prolog to
    jeden z popularniejszych języków programowania logicznego. Powstał on
    w celu automatycznej analizy języków naturalnych, jednakże jest on
    językiem ogólnego zastosowania i dobrze sprawdza się w programach 
    związanych ze sztuczną inteligencją. Program w Prologu składa się z faktów
    i reguł wnioskowania, aby go uruchomić należy wprowadzić odpowiednie 
    zapytanie. Prolog opiera się na rachunku predykatowym pierwszego rzędu,
    jednakże ogranicza się on do \textit{klauzul Horna}. Istnieją jednak
    wbudowane predykaty wyższego stopnia.

\section{Zakończenie}
	Zakończenie


\end{document}
